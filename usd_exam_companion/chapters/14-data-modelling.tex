\DomainHeader{Data Modelling}{13\%}

\begin{Objectives}
  \item Add a primvar to a mesh.
  \item Choose correct value types for attribute data.
  \item Represent custom metadata.
  \item Retrieve properties of a prim.
  \item Diagnose causes of unexpected visual results.
  \item Update \texttt{extent} after updating \texttt{points}.
\end{Objectives}

\begin{Resources}
  \item Learn OpenUSD: Scene Description Blueprints; Beyond the Basics
  \item OpenUSD user guide: Rendering — working with primvars
  \item OpenUSD API: Sdf datatypes; UsdGeomPrimvar; Boundable/Xformable; XformCache
\end{Resources}

\CheatSheetTitle
\begin{itemize}[leftmargin=*]
  \item \textbf{Attributes vs metadata}: attributes are time-sampleable properties; metadata describes prim/layer/stage behaviour.
  \item \textbf{Primvars}: namespaced as \texttt{primvars:*}; interpolation matters (constant/uniform/vertex/faceVarying).
  \item \textbf{UVs}: commonly \texttt{primvars:st} (often faceVarying).
  \item \textbf{extent}: bounding box; update it when \texttt{points} changes or downstream consumers may cull incorrectly.
\end{itemize}

\DrillsTitle
\subsection*{Drill 1: Primvar practice}
Write a minimal Mesh that includes \texttt{primvars:st} with faceVarying interpolation.
\AnswerLines{10}

\subsection*{Drill 2: Type discipline}
Give 6 examples of common types (token, rel, float3, color3f[], matrix4d, asset) and what you’d store in them.
\AnswerLines{8}
