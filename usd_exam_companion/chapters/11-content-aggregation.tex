\DomainHeader{Content Aggregation}{10\%}

\begin{Objectives}
  \item Add a new prototype to a \texttt{UsdGeomPointInstancer}.
  \item Edit an \texttt{instanceProxy} mesh (e.g., colour) without breaking instancing.
  \item Hide PointInstancer instances efficiently.
  \item Implement/manage USD instances for reuse in large scenes.
  \item Remove properties from instanced component prims in an assembly stage.
\end{Objectives}

\begin{Resources}
  \item Learn OpenUSD: Asset Modularity and Instancing
  \item OpenUSD API: Scenegraph Instancing; UsdGeomPointInstancer
\end{Resources}

\CheatSheetTitle
\begin{itemize}[leftmargin=*]
  \item \textbf{Two main mechanisms}: scenegraph instancing (native instancing) vs PointInstancer (many instances of prototypes).
  \item \textbf{PointInstancer}: instances are lightweight; per-instance transforms/visibility via arrays.
  \item \textbf{instanceProxy}: lets you target inside an instance without flattening it.
  \item \textbf{Don’t break instancing}: avoid authoring edits that force unique topology/material networks per instance unless intended.
  \item \textbf{Hiding instances}: usually via instance indices/visibility masks rather than duplicating geometry.
\end{itemize}

\DrillsTitle
\subsection*{Drill 1: “What breaks instancing?”}
List 5 edits that commonly de-instance things.
\AnswerLines{7}

\subsection*{Drill 2: InstanceProxy mental model}
Explain what path you expect when selecting a prim inside an instance.
\AnswerLines{6}

\subsection*{Mini-checklist}
\begin{itemize}[leftmargin=*]
  \item[$\square$] I can explain why instanceProxy exists.
  \item[$\square$] I can describe PointInstancer data layout (protoIndices + xforms).
\end{itemize}
