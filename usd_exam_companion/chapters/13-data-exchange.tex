\DomainHeader{Data Exchange}{15\%}

\begin{Objectives}
  \item Convert USD assets to common formats (e.g., glTF) and preserve fidelity.
  \item Document conceptual data mappings (e.g., USD $\rightarrow$ MaterialX).
  \item Explain USDC vs USDA trade-offs.
  \item Implement round-trip pipelines to/from a DCC.
  \item Use schemas for nonstandard attributes/structures during import/export.
  \item Write validators for exported USD assets.
  \item Write exporters/converters to USD.
  \item Write or extend USD importers in a DCC.
\end{Objectives}

\begin{Resources}
  \item Learn OpenUSD: Developing Data Exchange Pipelines
  \item OpenUSD tutorials: Traversing a Stage; Converting Between Layer Formats
  \item Tooling: \texttt{usdchecker}, \texttt{usdcat}, \texttt{usdzip}
\end{Resources}

\CheatSheetTitle
\begin{itemize}[leftmargin=*]
  \item \textbf{USDA}: readable, diff-friendly; heavier to load/parse.
  \item \textbf{USDC}: binary crate; faster IO; harder to diff manually.
  \item \textbf{Round-trip reality}: you will lose data unless you define mapping rules and validation.
  \item \textbf{Validation}: treat exports like “APIs” — validate schema presence, units, axes, bindings, and paths.
\end{itemize}

\DrillsTitle
\subsection*{Drill 1: USDA vs USDC choice}
Write 4 situations where you would mandate one over the other.
\AnswerLines{7}

\subsection*{Drill 2: Validator checklist}
List 10 checks your exporter should do before shipping an asset.
\AnswerLines{10}
