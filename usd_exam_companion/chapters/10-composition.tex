\DomainHeader{Composition}{23\%}

\begin{Objectives}
  \item Change the strength of an opinion.
  \item Choose an instancing style for different scales.
  \item Compare referencing vs payloads vs sublayers; identify use-cases.
  \item Reference the same animation with different time-offsets.
  \item Design layering strategies for multi-user workflows.
  \item Explain LIVERPS ordering.
  \item Decide when variants are/are not appropriate.
  \item Diagnose why an opinion does not take effect in a composed stage.
  \item Prepare internal assets for external delivery.
  \item Remove properties from instanced component prims in an assembly stage.
  \item Split a monolithic asset to support collaboration.
\end{Objectives}

\begin{Resources}
  \item Learn OpenUSD: Creating Composition Arcs
  \item OpenUSD tutorials: Referencing Layers; Authoring Variants
  \item OpenUSD glossary: LIVERPS, Sublayers, Variant Sets, References, Payloads, Layer Offsets, Edit Target, Flatten
  \item OpenUSD API: Scenegraph Instancing; UsdEditTarget; UsdStage::Flatten; UsdReferences
\end{Resources}

\CheatSheetTitle
\begin{itemize}[leftmargin=*]
  \item \textbf{Strength} in composition: stronger opinions override weaker ones.
  \item \textbf{LIVERPS} (common debugging lens): \textbf{L}ocal, \textbf{I}nherits, \textbf{V}ariants, \textbf{R}eferences, \textbf{P}ayload, \textbf{S}ublayers.
  \item \textbf{Sublayers}: stack whole layers; great for “edits on top”, shots, departments.
  \item \textbf{References}: bring in prim specs; good for asset assembly and reuse.
  \item \textbf{Payloads}: like references but can be deferred/unloaded for performance.
  \item \textbf{Variants}: discrete mutually exclusive choices; avoid using as a “parameter bag”.
  \item \textbf{Layer offsets}: time offset/scale on a reference for re-timing.
  \item \textbf{Debug habit}: ask “where is the strongest authored opinion and what arc brought it here?”
\end{itemize}

\DrillsTitle
\subsection*{Drill 1: LIVERPS in 60 seconds}
Given three layers (sublayer + reference + local override), state which wins for a single attribute and why.
\AnswerLines{6}

\subsection*{Drill 2: Reference vs Payload decision}
Write 3 criteria when you choose payload instead of reference.
\AnswerLines{6}

\subsection*{Drill 3: Variant misuse}
List 3 signs you are abusing variants (and what to use instead).
\AnswerLines{6}

\subsection*{Drill 4: Time-offset reasoning}
Explain how to reference the same animation twice with different start times.
\AnswerLines{6}

\subsection*{Mini-checklist (tick when mastered)}
\begin{itemize}[leftmargin=*]
  \item[$\square$] I can predict which layer wins for any single attribute.
  \item[$\square$] I can explain defaultPrim and why it matters for referencing.
  \item[$\square$] I can explain “why my override doesn’t take effect” without guessing.
\end{itemize}
